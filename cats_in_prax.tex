\documentclass{beamer}
\usepackage[T1]{fontenc}
\usepackage[utf8]{inputenc}
\usepackage{lmodern}
\usepackage{listings}


\definecolor{ingorange}{rgb}{1.0,0.38,0.0}
\useoutertheme{tree}
\setbeamercolor{titlelike}{fg=ingorange}
\setbeamertemplate{itemize item}{\color{ingorange}$\blacktriangleright$}

\definecolor{dkgreen}{rgb}{0,0.6,0}
\definecolor{gray}{rgb}{0.5,0.5,0.5}
\definecolor{mauve}{rgb}{0.58,0,0.82}

\lstdefinestyle{Scala}{
  frame=tb,
  language=scala,
  aboveskip=3mm,
  belowskip=3mm,
  showstringspaces=false,
  columns=flexible,
  basicstyle={\small\ttfamily},
  numbers=none,
  numberstyle=\tiny\color{gray},
  keywordstyle=\color{blue},
  commentstyle=\color{dkgreen},
  stringstyle=\color{mauve},
  frame=single,
  breaklines=true,
  breakatwhitespace=true,
  tabsize=3,
}

\title{Categories in the wild}
\author{Wouter Stekelenburg}

\newcommand\hide[1]{}

\begin{document}
\lstset{style=Scala}

\begin{frame}
  \titlepage
\end{frame}

\hide{
\begin{frame}
\frametitle{Categories are like \dots}
Categories are like cats. It is easy to learn what they are, but that doesn't mean you understand them or can work with them without getting hurt.
\end{frame}
}

\section{Set up}
\begin{frame}[fragile]
\frametitle{Naked interface}

\begin{lstlisting}
trait Cassandra {
  def prepare(statement: String): PreparedStatement
  def prepare(statement: RegularStatement): PreparedStatement
  def execute(statement: String): Unit
  def execute(statement: Statement): Unit
}
\end{lstlisting}

\end{frame}

\begin{frame}[fragile]
\frametitle{Wrapped interface}

\begin{lstlisting}
trait Cassandra {
  def prepare(statement: String): F[PreparedStatement]
  def prepare(statement: RegularStatement): F[PreparedStatement]
  def execute(statement: String): F[Unit]
  def execute(statement: Statement): F[Unit]
}
\end{lstlisting}

\end{frame}

\section{Categorical specifications}

\begin{frame}[fragile]
\frametitle{Map}
\begin{lstlisting}
map: (X => Y) => F[X] => F[Y]

// map((x: X) => x)(y) === y
// map(f)(map(g)(y)) === map((x: X) => f(g(x)))(y)
\end{lstlisting}
\end{frame}

\begin{frame}
\frametitle{Functor}
\includegraphics{funktor.jpeg}
\end{frame}

\begin{frame}[fragile]
\frametitle{Zip}
\begin{lstlisting}
zip: (F[X],F[Y]) => F[(X,Y)]

// map(_._1)(zip(fx, fy)) === fx
// map(_._2)(zip(fx, fy)) === fy
// zip(map(_._1)(z),map(_._2)(z)) === z
\end{lstlisting}
\end{frame}

\begin{frame}[fragile]
\frametitle{Unit}
\begin{lstlisting}
unit: X => F[X]

// map(f)(unit(x)) === unit(f(x))
\end{lstlisting}
\end{frame}

\begin{frame}[fragile]
\frametitle{Traverse}
\begin{lstlisting}
traverse: (X => F[X]) => List[X] => F[List[X]]

// traverse(f)(Nil) === unit(Nil)
// traverse(f)(h :: t) === map{
//   case (x,y) => x :: y
// }(zip(f(h),traverse(f)(t)))
\end{lstlisting}
\end{frame}

\begin{frame}[fragile]
\frametitle{Applicative functor}
\begin{lstlisting}
map: (X => Y) => F[X] => F[Y]
zip: (F[X],F[Y]) => F[(X,Y)]
unit: X => F[X]
traverse: (X => F[X]) => List[X] => F[List[X]]
\end{lstlisting}
\end{frame}

\begin{frame}[fragile]
\frametitle{Bind}
\begin{lstlisting}
bind: (X => F[Y]) => F[X] => F[Y]

// bind(f)(bind(g)(x)) === bind((y: X) => bind(f)(g(y))(x))
// bind(unit)(x) === x
// bind(f)(unit(x)) === unit(f(x))

// map(f)(x) === bind((y: X) => unit(f(y)))(x)
// zip(x,y) === bind((x0: X) => map((y0: Y) => (x0, y0)))
\end{lstlisting}
\end{frame}

\begin{frame}[fragile]
\frametitle{ReduceM}
\begin{lstlisting}
reduceM: F[X] => (X => X => F[X]) => List[X] => F[X]

// reduceM(a)(b)(Nil) === unit(a)
// reduceM(a)(b)(h :: t) === bind(b(h))(reduceM(a)(b)(t))
\end{lstlisting}
\end{frame}

\begin{frame}[fragile]
\frametitle{Monad}
\begin{lstlisting}
unit: X => F[X]
bind: (X => F[Y]) => F[X] => F[Y]
reduceM: F[X] => ((X,X) => F[X]) => List[X] => F[X]
\end{lstlisting}
\end{frame}


\section{Implementations}

\subsection{scala.concurrent.Future}
\begin{frame}
\frametitle{scala.concurrent.Future}
\begin{itemize}
\item used in calendars api, savings me api
\item Scala standard library
\end{itemize}
\end{frame}

\begin{frame}[fragile]
\frametitle{Asynchronous interface}
\begin{lstlisting}
abstract class Cassandra(implicit ec:ExecutionContext) {
  def prepare(statement: String): Future[PreparedStatement]
  def prepare(statement: RegularStatement): Future[PreparedStatement]
  def execute(statement: String): Future[Unit]
  def execute(statement: BoundStatement): Future[Unit]
}
\end{lstlisting}
\begin{itemize}
\item \verb|Future.successful| and \verb|Future.apply| for \verb|unit|
\item \verb|flatMap| for \verb|bind|
\item \verb|Future.traverse| exists
\end{itemize}
\end{frame}

\subsection{Continuation passing style}
\begin{frame}
\frametitle{Continuation passing style}
Continuation passing style
\begin{itemize}
\item abstracts over the notion of registering callbacks
\item can mimic any monad
\item reasonable solution in Java
\end{itemize}
\end{frame}

\begin{frame}[fragile]
\frametitle{Continuation passing style monad}
\begin{lstlisting}
object CPSMonad {

  type CPS[X] = (X => Unit) => Unit

  def unit[X](x: => X): CPS[X] = (c: X => Unit) => c(x)

  def bind[X, Y](f: X => CPS[Y])(cx: CPS[X]): CPS[Y] = (cy: Y => Unit) => cx((x: X) => f(x)(cy))
  
  //...
}
\end{lstlisting}
\end{frame}
\begin{frame}[fragile]
\frametitle{Stacksafe traverse}
\begin{lstlisting}
  def traverse[X, Y](f: X => CPS[Y])(lx: List[X]): CPS[List[Y]] = {
    @tailrec def helper(lx: List[X], cy: List[Y] => Unit): Unit = lx match {
      case Nil => cy(Nil)
      case hx :: tx => helper(tx, (ty: List[Y]) => f(hx)((hy: Y) => cy(hy :: ty)))
    }
    helper(lx,_)
  }
\end{lstlisting}
\end{frame}

\subsection{Free monads}
\begin{frame}
\frametitle{Free monads}
\begin{itemize}
\item Reify everything
\item Build and manipulate ASTs for later interpretation
\end{itemize}
\end{frame}

\begin{frame}[fragile]
\frametitle{Free monad constructors}
\begin{lstlisting}
sealed trait Free[F[_], X]

case class Return[F[_], X](x: X) extends Free[F, X]

case class Effect[F[_], X](fx: F[X]) extends Free[F, X]

case class Bind[F[_], X](f: X => Free[F, X], frx: Free[F, X]) extends Free[F, X]

case class Traverse[F[_], X](f: X => Free[F, X], lx: List[X]) extends Free[F, List[X]]
\end{lstlisting}
\end{frame}

\begin{frame}[fragile]
\frametitle{Effect constructors}
\begin{lstlisting}
sealed trait CassandraEffect[X]

case class PrepareString(statement: String) extends CassandraEffect[PreparedStatement]

case class PrepareRegular(statement: RegularStatement) extends CassandraEffect[PreparedStatement]

case class ExecuteString(statement: String) extends CassandraEffect[ResultSet]

case class ExecuteStatement(statement: Statement) extends CassandraEffect[ResultSet]
\end{lstlisting}
\end{frame}

\subsection{Generic monad}
\begin{frame}
\frametitle{Generic monad}
\begin{itemize}
\item abstract over monads themselves
\item dependency injection applied to monads
\end{itemize}
\end{frame}

\begin{frame}[fragile]
\frametitle{Monad type class}
\begin{lstlisting}
trait Monad[M[_]] {

  def unit[X](x: X): M[X]

  def bind[X, Y](f: X => M[Y])(mx: M[X]): M[Y]
  
  def traverse[X, Y](f: X => M[Y])(lx: => List[X]): M[List[X]]
}
\end{lstlisting}
\end{frame}

\begin{frame}[fragile]
\frametitle{Generically wrapped interface}
\begin{lstlisting}
abstract class Cassandra[F: Monad] {
  def prepare(statement: String): F[PreparedStatement]
  def prepare(statement: RegularStatement): F[PreparedStatement]
  def execute(statement: String): F[Unit]
  def execute(statement: Statement): F[Unit]
}
\end{lstlisting}
\end{frame}

\section{Conclusion}

\begin{frame}
\frametitle{Summary}
\begin{itemize}
\item `functor', `applicative functor' and `monad' specify interfaces that handle callbacks, in increasing power
\item `Future', `continuation passing style', `free monads' and `generic monads' show a range of technical implementations that can hide behind those interfaces
\end{itemize}

\end{frame}

\end{document}
